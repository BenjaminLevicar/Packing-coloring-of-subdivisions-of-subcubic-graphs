\documentclass[11pt,a4paper]{article}

\usepackage[utf8]{inputenc}
\usepackage[T1]{fontenc}
\usepackage[slovene]{babel}
\usepackage{amsmath}
\usepackage{amssymb}

\begin{document}

\title{Pakirno kromatično barvanje subdivizij subkubičnih grafov}
\author{Benjamin Levičar, Jan Nabergoj}

\maketitle

\section{Opis problema}

Preveriti  želimo domnevo: \emph{Za vsako subdivizijo subkubičnega grafa $G$ velja $\chi_\rho(S(G)) \leq 5$}. Tega se bomo lotili z iskanjem protiprimera, torej želimo poiskati subdivizijo subkubičnega grafa, ki ima pakirno kromatično število vsaj 6.

Razložimo najprej pojme iz domneve:
\begin{itemize}
    \item \emph{Subdivizija $S(G)$ grafa $G$} je graf dobljen tako, da namesto vsake povezave v grafu $G$ postavimo vozlišče in ga povežemo z vozliščema,
          ki sta bila krajišči odstranjene povezave. Nova vozlišča so tako stopnje $2$.
    \item \emph{Subkubičen graf} je graf v katerem imajo vozlišča stopnjo največ $3$.
    \item \emph{Pakirno kromatično barvanje} je barvanje grafa $G$ z barvami $1, 2, \dots$ za katerega velja, da za poljubni vozlišči $u, v$ barve $i$ velja $d(u, v) > i$.
          Najmanjše število barv potrebnih za to imenujemo \emph{Pakirno kromatično število} grafa, označimo ga $\chi_\rho(G)$.
\end{itemize}


\section{Potek dela}

Najprej bomo implementirali celoštevilski linearni program, ki nam bo poiskal pakirmo kromatično število danega grafa. 

\begin{enumerate}
    \item Domnevo bomo najprej sistematično preverili za vse manjše grafe (recimo do 15 vozlišč). Torej zgenerirali bomo vse grave in na njihovih subdivizijah pognali CLP za iskanje pakirnega kromatičnega števila.
    \item Za večje grafe bomo uporabili metodo stohastičnega iskanja. To storimo saj je grafov z več vozližči zelo veliko in bi nam sistematično iskanje vzelo preveč časa. Grafe bomo generirali naključno in iskali njihova pakirna kromatična števila.
    \item Dobljene rezultate iz prejšnjih točk bomo analizirali in poskušali poiskati kakšne skupne lastnosti grafov z največjim pakirnom kromatičnim številom. Nato bomo znova uporabili stohastično iskanje na grafih z enakimi lastnostmi.
\end{enumerate}

Da bo iskanje bolj učinkvito bomo pobrskali po literaturi in poiskati kakšne lastnosti grafov, ki nam zagotovijo, da bo pakirno kromatično število majhno
oz. manjše od $5$.




\end{document}
