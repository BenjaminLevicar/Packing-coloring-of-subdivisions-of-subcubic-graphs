\documentclass[11pt,a4paper]{article}

\usepackage[utf8]{inputenc}
\usepackage[T1]{fontenc}
\usepackage[slovene]{babel}
\usepackage{amsmath}
\usepackage{amssymb}

\begin{document}

\title{Naslov}
\author{Benjamin Levičar, Jan Nabergoj}

\maketitle

\section{Opis problema}

Preveriti  želimo domnevo: \emph{za vsako subdivizijo subkubičnega grafa $G$ velja $\chi_\rho(S(G)) \leq 5$}. Poskušali bomo poiskati protiprimer.

Razložimo najprej pojme iz domneve:
\begin{itemize}
    \item \emph{Subdivizija $S(G)$ grafa $G$} je graf dobljen tako, da namesto vsake povezave v grafu $G$ postavimo vozlišče in ga povežemo z vozliščema,
          ki sta bila krajišči odstranjene povezave. Nova vozlišča so tako stopnje $2$.
    \item \emph{Subkubičen graf} je graf v katerem imajo vozlišča stopnjo največ $3$.
    \item \emph{Packing coloring} je barvanje grafa $G$ z barvami $1, 2, \dots$ za katerega velja, da za poljubni vozlišči $u, v$ barve $i$ velja $d(u, v) > i$.
          Najmanjše število barv potrebnih za to imenujemo \emph{Packing coloring number} grafa, označimo ga $\chi_\rho(G)$.
\end{itemize}


\section{Potek dela}

Najprej bomo implementirali celoštevilski linearni program, ki nam bo poiskal packing coloring number danega grafa. 

\begin{enumerate}
    \item Domnevo bomo preverili za vse manjše grafe (recimo do 15 vozlišč), tako da jih vse zgeneriramo in poženemo CLP na njihovih subdivizijah.
    \item Za večje grafe bomo uporabili stohastično iskanje, torej bomo naključno generirali grafe in iskali njihova Packing coloring number.
    \item Dobljene rezultate iz prejšnjih točk bomo analizirali in poskušali poiskati kakšne skupne lastnosti grafov z največjim Packing coloring
          number. Nato bomo znova uporabili stohastično iskanje na grafih z enakimi lastnostmi.
\end{enumerate}

Da bo iskanje bolj učinkvito bomo pobrskali po literaturi in poiskati kakšne lastnosti grafov, ki nam zagotovijo, da bo Packing coloring number majhno
oz. manjše od $5$.




\end{document}